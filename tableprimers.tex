%=================================== PhD setting
\documentclass[12pt]{scrreprt} %paraskip changes, {scrreprt} for thesis
\usepackage{courier} % for courier fonts
\usepackage{mathptmx} % for roman text fonts or {times}
\usepackage{graphicx} % also to resize the tables to textwidth, {graphics} is older
\usepackage{float} % to prevent floating with H
\usepackage{booktabs} % for nice looking tables
\usepackage{tablefootnote} % for tablefootnotes see documentation for comparisons w/ alternatives
\usepackage{tabularx} % to set table width to textwidth
\usepackage{colortbl} % for shading tables
\usepackage{hyperref}
\usepackage{xcolor} %to change color of the different links in latex
%\usepackage{geometry}
%\geometry{a4paper,left=15mm,right=15mm, top=1cm, bottom=2cm}
\definecolor{dark-red}{rgb}{0.4,0.15,0.15}
\definecolor{dark-blue}{rgb}{0.15,0.15,0.4}
\definecolor{medium-blue}{rgb}{0,0,0.5}
\hypersetup{
    colorlinks, linkcolor={gray},
    citecolor={dark-blue}, urlcolor={medium-blue}
}

\usepackage[left=1.5in, right=1in, top=1in, bottom=1in, includefoot, headheight=13.6pt]{geometry}% http://codeinthehole.com/writing/writing-a-thesis-in-latex/
%todo: geometry is not part of KOMA, but when omitted create a footnote before table
\begin{document}
\begin{table}[htbp]
%\caption{All primers which were used in RT-qPCR experiments.} % this will cause a problem
%todo: try to fix the caption, when inserted the footnotes are shifted before table
\label{tab:primers} % use for cross-reference
\resizebox{\textwidth}{!}{ %
\begin{tabular}{rllclllc} % 8 columns
\toprule
\textbf{Transcript} & \textbf{Primer sequence\tablefootnote{Letters in bold refer to the spanning part of the sequence}} & \textbf{RefSeq\tablefootnote{RefSeq: NCBI database of reference sequences http://www.ncbi.nlm.nih.gov}} & \textbf{AnT\tablefootnote{AnT: annealing temperature of the qPCR reaction ($ 95^{15} $, $ 94^{\prime 15} $, $ AnT^{\prime 30} $, $ 72^{\prime30} $, $ 72^5 $)}} & \textbf{Amplicon} & \textbf{Exons\tablefootnote{Indicates exons flanked (f) or spanned (s) by a primer set followed by the total number of exons for that transcript, e.g.\textit{ACTB} primer set flanks exon 3 and exon 4 - total exons are 6}} & \textbf{Intron} & \textbf{\textit{E}\tablefootnote{Amplification efficiency as determined by the dilution curve}} \\ 
& \textbf{(from $ 5^{'} $ to $ 3^{'} $)} &  & \textbf{($^{\circ} C$)} & \textbf{size (bp)} & & \textbf{size (bp)} & \\ 
\hline \multicolumn{1}{l}{\textit{\textbf{ACTB}}} & \multicolumn{7}{l}{a reference gene that encodes a cytoskeletal actin}\\ 
\ fwd & \texttt{AAGGCCAACCGCGAGAAGATG} & NM\_001101 & 57 & 100 & f(3,4)-6 & 441 & 2 \\ 
rev & \texttt{CAGAGGCGTACAGGGATAGCAC} &  &  &  &  &  &  \\ 
\hline \multicolumn{1}{l}{\textit{\textbf{RPL29}}} & \multicolumn{7}{l}{a reference gene that encodes a ribosomal protein in the 60S subunit} \\ 
\ fwd & \texttt{GCCAAGTCCAAGAACCACAC} & NM\_000992 & 57 & 133 & f(2,4)-4 & 1233 & 2 \\ 
\ rev & \texttt{CAAAGCGCATGTTCCTCAGG} &  &  &  &  &  &  \\ 
\hline \multicolumn{1}{l}{\textit{\textbf{B2M}}} & \multicolumn{7}{l}{\textit{$ \beta $2-microglobulin}, a reference gene that encodes a protein associated with MHC\tablefootnote{The major histocompatibility complex}} \\ 
\ fwd & \texttt{GAGTATGCCTGCCGTGTGAAC} & NM\_004048 & 57 & 100 & s(2,\textbf{4})-4 & 1877 & 1.95 \\ 
\ rev & \texttt{CGGCATCTTCAAAC\textbf{CTCCATG}} & & &  &  &  &  \\ 
\hline \multicolumn{1}{l}{\textit{\textbf{IL8}}} & \multicolumn{7}{l}{encodes a protein in the CXC chemokine family called IL-8} \\ 
\ fwd & \texttt{CTTCCTGATTTCTGCAGCTCTG} & NM\_000584 & 57 & 193 & f(1,3)-4 & 1090 & 2 \\ 
\ rev & \texttt{GAGCTCTCTTCCATCAGAAAGC} &  &  &  &  &  &  \\ 
\hline \multicolumn{1}{l}{\textit{\textbf{IL17A}}} & \multicolumn{7}{l}{one of six transcripts (A-F) that encode a proinflammatory cytokine produced by activated T cells} \\ 
\ fwd & \texttt{\textbf{GGAATCTCCA}CCGCAATGAG} & NM\_002190 & 57 & 104 & s(\textbf{2},3)-3 & 1249 & 2 \\ 
\ rev & \texttt{GTAGTCCACGTTCCCATCAG} &  &  &  &  &  &  \\ 
\hline \multicolumn{1}{l}{\textit{\textbf{IL23A}}} & \multicolumn{7}{l}{encodes p19    that pairs with p40 to form IL-23} \\ 
\ fwd & \texttt{\textbf{GACACAT}GGATCTAAGAGAAGAG} & NM\_016584 & 57 & 109  & s(\textbf{1},\textbf{3})-4 & 385 & 1.95 \\ 
\ rev & \texttt{\textbf{AA}CTGACTGTTGTCCCTGAG} &  &  &  &  &  &  \\ 
\hline \multicolumn{1}{l}{\textit{\textbf{IL23R}}} & \multicolumn{7}{l}{encodes IL23A receptor which is required, along with IL12RB1, for signaling of IL-23 via p19} \\ 
\ fwd & \texttt{TCCTGTGAAATGAGATACAAGGC} & NM\_144701 & 59 & 104 & f(6,7)-11 & 12518 & 1.95 \\ 
\ rev & \texttt{GGCTCCAAGTAGAATTCTGACTG} &  &  &  &  &  &  \\ 
\hline \multicolumn{1}{l}{\textit{\textbf{EBI3}}} & \multicolumn{7}{l}{\textit{IL27B}, encodes Ebi3 that pairs with p28 to form IL-27} \\ 
\ fwd & \texttt{AGCTTCGTGCCTTTCATAACAG} & NM\_005755 & 57 & 140 & f(3,4)-5 & 1359 & 2 \\ 
\ rev & \texttt{AGTGAGAAGATCTCTGGGAAGG} &  &  &  &  &  &  \\ 
\hline \multicolumn{1}{l}{\textit{\textbf{gp130}}} & \multicolumn{7}{l}{\textit{IL6ST}, \textit{oncostatin M receptor}, \textit{CD130}, encodes a signal transducer that is shared by many cytokines} \\ 
\ fwd & \texttt{ACTGTCCAAGACCTTAAACCT} & NM\_001190981 & 57 & 145 & f(7,8)-17 & 3110 & 1.98 \\ 
\ rev & \texttt{AGAAACTTGGTGCTTTAGATGG} &  &  &  &  &  &  \\ 
\hline \multicolumn{1}{l}{\textit{\textbf{IL12A}}} & \multicolumn{7}{l}{\textit{NKSF1}, \textit{CLMF1}, encodes p35 that pairs with p40 and Ebi3 to form IL-12 and IL-35, respectively} \\ 
\ fwd & \texttt{AATGTTCCCATGCCTTCACC} & NM\_000882 & 59 & 110 & f(2,3)-7 & 2699 & 2 \\ 
\ rev & \texttt{CAATCTCTTCAGAAGTGCAAGGG} &  &  &  &  &  &  \\ 
\hline \multicolumn{1}{l}{\textit{\textbf{IL12RB2}}} & \multicolumn{7}{l}{encodes a protein that forms along with IL12RB1 a high affinity receptor for IL-12 via p35} \\ 
\ fwd & \texttt{CCTCTTCACTTCCATCCACATTC} & NM\_001559 & 57 & 137 & f(5,6)-16 & 1202 & 1.96 \\ 
\ rev & \texttt{AAGCAGTACCAGTCCCTCATC} &  &  &  &  &  &  \\ 
\hline \multicolumn{1}{l}{\textit{\textbf{IL12B}}} & \multicolumn{7}{l}{\textit{NKSF2}, \textit{CLMF2}, encodes p40 that pairs with p35 and p19 to form IL-12 and IL-23, respectively} \\ 
\ fwd & \texttt{\textbf{GGACATCA}TCAAACCTGACC} & NM\_002187 & 56 & 123 & s(\textbf{5},6)-8 & 1412 & 2 \\ 
\ rev & \texttt{AGGGAGAAGTAGGAATGTGG} &  &  &  &  &  &  \\ 
\hline \multicolumn{1}{l}{\textit{\textbf{IL12RB1}}} & \multicolumn{7}{l}{encodes a low affinity protein that binds to IL-12 and, along with IL12RB2, forms IL-12 receptor} \\ 
\ fwd & \texttt{GCTGTACACTGTCACACTCTG} & NM\_005535 & 57 & 132 & f(4,5)-17 & 3176 & 1.9 \\ 
\ rev & \texttt{AACTTGGACACCTTGATGTCTC} &  &  &  &  &  &  \\ 
\hline \multicolumn{1}{l}{\textit{\textbf{IL27A}}} & \multicolumn{7}{l}{\textit{IL30}, encodes p28 that pairs with Ebi3 to form IL-27} \\ 
\ fwd & \texttt{CTCCCTGATGTTTCCCTGAC} & NM\_145659 & 57 & 145 & f(3,4)-5 & 1560 & 1.95 \\ 
\ rev & \texttt{TCCTCTCCATGTTGGTCCAG} &  &  &  &  &  &  \\ 
\hline \multicolumn{1}{l}{\textit{\textbf{IL27RA}}} & \multicolumn{7}{l}{\textit{WSX1}, \textit{CRL1}, \textit{TCCR}, \textit{zcytor1}, encodes a glycosylated transmembrane protein} \\ 
\ fwd & \texttt{AGATGTGTGGGTATCAGGGAAC} & NM\_004843 & 57 & 102 & f(6,7)-14 & 3364 & 2 \\ 
\ rev & \texttt{ACTTTGTAGCTCACCTGCAC} &  &  &  &  &  &  \\ 
%\hline &  &  &  &  &  &  &  \\ 
%\hline
\bottomrule
\end{tabular} }
\end{table}
\newpage
\ and a second page here. this is a version control check here \ldots.
\end{document}
