\documentclass[oneside]{scrbook} % you can use {scrreprt} alternatively, you can add toc=graduated, or paraskip change space between paragraphs 
\title{Ph.D. Thesis}
\author{Fadi Fouad Al-Sammak}
\date{January 2014}
\titlehead{A Thesis submitted for the degree of Doctor of Philosophy}
\publishers{Faculty of Natural Sciences\\Otto von Guericke University Magdeburg}
%\setcounter{errorcontextlines}{999} % to see more of the errors in latex
%todo: change the dimensions according to OvGU Magdeburg
%\usepackage[left=1.5in, right=1in, top=1in, bottom=1in, includefoot, headheight=13.6pt]{geometry}% http://codeinthehole.com/writing/writing-a-thesis-in-latex/, don't use geometry with KOMO-script class of file like this
%===================== Debugging file ===========================
\usepackage{tabu} %misc
\setcounter{errorcontextlines}{999} % to see more of the errors in latex
\usepackage{layouts} % to show the layout of the document there is {layouts} with \pagevalues
%\usepackage[texcoord,gridunit=cm]{showframe} %show layout of document 
%=====================Language, Fonts============================
\usepackage{csquotes} % required by babel,[strict] will turn warnings into errors
\usepackage[ngerman,english]{babel} %last language is the current
\selectlanguage{english}
\usepackage{libertine} % must be with XeLaTeX not PdfLaTeX
%\usepackage{courier,mathptmx,amsmath,amsfonts,amssymb} %fonts
\usepackage{courier,amsmath,amsfonts} %fonts
%\usepackage[utf8]{inputenc} % coding, LaTeX for complete novices font, should be before {biblatex}
%\usepackage[math]{anttor} % LaTeX for complete novices font
%\usepackage{libris} % LaTeX for complete novices font
%\renewcommand*{\ttdefault}{txtt} % LaTeX for complete novices font
%\usepackage[T1]{fontenc} % allows italics in titles
%\usepackage[sc]{mathpazo} % Palatino
%\linespread{1.05} % Palatino needs more leading (space between lines)
%\usepackage{cmbright}
%=====================Hyphenation issues ============================
\usepackage[none]{hyphenat}% prevent hyphenation or line breakups
\usepackage[protrusion = true,final]{microtype} % handle the overfull hboxes in pdflatex only, should be after fonts!
\emergencystretch=1.5em % crucial to get rid of rare cases of overfull hboxes, very handy indeed
%=====================Floats ============================
%\usepackage{flafter} % to ensure the figure appears after being referenced in text
\usepackage{floatrow} % for placing of captions use with {caption} package	
%\floatsetup[table]{style=plaintop} % keys from table 11 manual
\usepackage{siunitx} % to align numbers by their decimal points and SI units
%\floatsetup[figure]{style=plain} % keys from table 11 manual
\usepackage[labelsep=period]{caption} % for appearance of captions not placings
%\usepackage{subcaption} % incompatible with subfig, or subfigure (obsolete)
\usepackage{booktabs,colortbl} %tables
\usepackage{float} % H parameter will force the float in its place
%\usepackage{tabularx} % to set table width to textwidth
\usepackage{tabulary} % to set table width to textwidth
\usepackage{graphicx,xcolor} %graphcis
%\usepackage{eso-pic} % to add a logo anywhere you want
%\usepackage[percent]{overpic} % over pic text, can be replaced by tikz node
%=====================Referencing============================
\usepackage{hyperref} % referencing make sure you load the {xcolor} package, keep this package at last, one exception is the glossaries package, should be after hyperref
\definecolor{dark-red}{rgb}{0.4,0.15,0.15}
\definecolor{dark-blue}{rgb}{0.15,0.15,0.4}
\definecolor{medium-blue}{rgb}{0,0,0.5}
\hypersetup{
    colorlinks, linkcolor={dark-red},
    citecolor={dark-blue}, urlcolor={medium-blue} %url can be magenta 
    }
%=============================== Glossaries/Acronyms/Notations/Index===
\usepackage{datagidx}
\newgidx{index}{Index}
\newgidx{glossary}{Glossary}
\newgidx{acronym}{Acronyms}
\newgidx{notation}{Notation}
\DTLgidxSetDefaultDB{glossary}
\newterm [%
description={a rectangular table of elements},% brief description
plural={matrices}% the plural
]%
{matrix}% the name
\DTLgidxSetDefaultDB{acronym}
\newacro{svm}{support vector machine}
\DTLgidxSetDefaultDB{notation}
\newterm
[%
label={not:set},% label
description={A set},%
sort={S}%
]%
{\ensuremath{\mathcal{S}}}
\DTLgidxSetDefaultDB{index}
\newterm
[%
label={function},%
text={function}%
]%
{functions}
\newterm
[%
see={sqrt},%
]%
{square root}
\newterm
[%
label={fn.sqrt},
parent={function}
]%
{\texttt{sqrt()}}
\newterm
[%
label={sqrt},
]%
{sqrt()}
\newterm{tautology}
\newterm{contradiction}

%\usepackage{makeidx} % to make index see the new command, F6 -> F12 -> F6  
%\makeindex % to make index related to makeidx pkg add \printindex at the desired place at the end of the document
%===============================Bibliography or References ===============
\usepackage[backend=biber,style=chem-acs,sorting=none,isbn=false,doi=false]{biblatex} % should be after {inputenc}, you can use {showkeys} to show keys
%\usepackage{csquotes} % Recommended for biblatex along with {babel}
%\addbibresource{F:phd/ref/Exported Items.bib} % here you add the extension
\addbibresource{C:/Users/FADI/Dropbox/phd/ref/Exported Items.bib} % here you add the extension
\ExecuteBibliographyOptions{%
  citetracker=true,% Citation tracker enabled in order not to repeat citations, and have two lists.
  sorting=none,% Don't sort, just print in the order of citation
  alldates=long,% Long dates, so we can tweak them at will afterwards
  dateabbrev=false,% Remove abbreviations in dates, for same reason as ``alldates=long''
  articletitle=true,% To have article titles in full bibliography
  maxcitenames=999% Number of names before replacing with et al. Here, everyone.
  }

% No brackets around the number of each bibliography entry
\DeclareFieldFormat{labelnumberwidth}{#1\addperiod}

% Suppress article title, doi, url, etc. in citations
\AtEveryCitekey{%
  \ifentrytype{article}
    {\clearfield{title}}
    {}%
  \clearfield{doi}%
  \clearfield{url}%
  \clearlist{publisher}%
  \clearlist{location}%
  \clearfield{note}%
}

% Print year instead of date, when available; make use of urldate
\DeclareFieldFormat{urldate}{\bibstring{urlseen}\space#1}
\renewbibmacro*{date}{% Based on date bib macro from chem-acs.bbx
  \iffieldundef{year}
    {\ifentrytype{online}
       {\printtext[urldate]{\printurldate}}
       {\printtext[date]{\printdate}}}
    {\printfield[date]{year}}}

% Remove period from titles
\DeclareFieldFormat*{title}{#1}
\DeclareFieldFormat[book]{date}{\textbf{#1}} % Fadi added this line
% Embed doi and url in titles, when available
\renewbibmacro*{title}{% Based on title bib macro from biblatex.def
  \ifboolexpr{ test {\iffieldundef{title}}
               and test {\iffieldundef{subtitle}} }
    {}
    {\ifboolexpr{ test {\ifhyperref}
                  and not test {\iffieldundef{doi}} }
       {\href{http://dx.doi.org/\thefield{doi}}
          {\printtext[title]{%
             \printfield[titlecase]{title}%
             \setunit{\subtitlepunct}%
             \printfield[titlecase]{subtitle}}}}
       {\ifboolexpr{ test {\ifhyperref}
                     and not test {\iffieldundef{url}} }
         {\href{\thefield{url}}
            {\printtext[title]{%
               \printfield[titlecase]{title}%
               \setunit{\subtitlepunct}%
               \printfield[titlecase]{subtitle}}}}
         {\printtext[title]{%
            \printfield[titlecase]{title}%
            \setunit{\subtitlepunct}%
            \printfield[titlecase]{subtitle}}}}%
     \newunit}%
  \printfield{titleaddon}%
  \clearfield{doi}%
  \clearfield{url}%
  \clearlist{language}%
  \clearfield{note}%
  \ifentrytype{article}% Delimit article and journal titles with a period
    {\adddot}
    {}}

%=====================Renew commands==========================
%=====================New commands============================
\newcommand{\species}[2]{\emph{#1 #2}} % for any species name
\newcommand{\pcr}{Real-Time Quantitative Polymerase Chain Reaction } % Note that one space ws added at the end of the command.
\newcommand*{\keyword}[1]{{\color{blue} #1} \index{#1}} % related to makeidx package starred to track down error easily when braces are forgotten
%=====================Include only ===========================
%\includeonly{} % comma separated file list, alternative there is {excludeonly}, this applies only to \include{} but not to \input{}, the former creates a new page thru \clearpage command
%todo: add all packages inside {} according to categories
\begin{document}
\maketitle
\frontmatter
\tableofcontents
\listoffigures
\listoftables
\chapter{Acknowledgements}
I would like to thank my supervisor, Professor Someone. This
research was funded by the Imaginary Research Council.
\chapter{Abstract}
A brief summary of the project goes here. Using \pcr we analysed the samples.
% A glossary and list of acronyms may go here
% or may go in the back matter.
\mainmatter
%\pagevalues % comment out if not needed; {layout}
%\layout % comment out if drawing the layout is not needed; {layout}
%\pagestyle{empty}\mbox{} % {showframe} layout overlaid on text; comment out 
%====================Introduction=============================
\chapter{Introduction}
\label{ch:intro}
\texttt{This is an example document.}\\
Out of \num{12890} experiments, \num{1289} of them had a mean
squared error of \num{.346} and \num{128} of them had a mean
squared error of \num{1.23e-6}.\\ 
Fadi is now busy typing his thesis using LaTeX program \index{program}.\par
Also he is planning to learn more of it! Although it is a new one, however it is pleasing to the eye.\par
I did some changes here for git lesson and you can refer to~\ref{ch:intro}\\
% leaving one space means new paragraph
\LaTeX\ {FADI.}\footnote{the author of this document}
this document was created on \today\\
%\texttt{Fadi is writing a LaTeX document}\\
%\textrm{Fatenisnow typing in the daily light} 
%\textsc{\LargeDoctoralThesis}\\[0.5cm] % Thesis type
This is to change the git commit click~\url{https://github.com/doctorate}
This is just a example of the  language tool that has be already installed \\
This is a way to use verbatim text \verb*|usepackage{  }| in some cases.  
This is a example input to to show you how LanguageTool works or to what extend.
A in context to  tour in London would be nice. I want to use special characters like $\gamma$ and $\beta$ for example.\\
\[ \lim_{x\to\infty} f(x) \]
$ x -y $

\ I want to do citation here~\parencite{becker_cutting_2006} using the \verb|biblatex| package with \verb|biber| backend and TexStudio. You can also use the~\textcite{becker_cutting_2006}. You can also use the~\parencite{hebenstreit_analysis_2011}. Also see~\parencite{wickham_ggplot2:_2009}. \parencite{matthias_kohl_slqpcr:_2007}.
see~\parencite{kennedy_pcr_2011}.
see~\parencite{vignali_il-12_2012}.
\parencite{akira_role_2000}. Multiple references in \parencites(compare)(){aihara_mechanisms_1997}[34]{aggarwal_interleukin-23_2003}

\section{I am considerate
\protect\footnote{and protect my footnotes}}
\begin{table}[H]
\begin{tabular}{SSS}% syntax for siunitx v2; for v1 use "tabformat"
55.5 & 44.5 & 23.3 \\
7.77 & 55.666 & 88.999\\	
99.9 & 123.1223 & 44.777 \\
\end{tabular}
\end{table}
\par
% !TeX spellcheck = en_GB
Hiermit erkl\"are ich, dass ich die vorliegende Arbeit selbst\"andig verfasst und keine anderen als die angegebenen Quellen und Hilfsmittel verwendet habe.\\
\ Magdeburg, Stra\ss e den\ldots\ldots
\chapter{Materials and Method}
%This is to index \index{program} program again in a different page. \index{adz}. \index{OAEolian@\AE olian} \index{sqrt()@\texttt{sqrt()}}. \index{sqrt()@\texttt{sqrt()}|textbf},\index{eigenvector|seealso{eigenvalue}}
%\label{ch:m&m}
\Glspl{matrix} are usually denoted by a bold capital letter, such
as $\mathbf{A}$. The \gls{matrix}'s $(i,j)$th element is usually
denoted $a_{ij}$. \Gls{matrix} $\mathbf{I}$ is the identity
\gls{matrix}.
First use: \acr{svm}\@. Next use: \acr{svm}\@. Full: \gls{svm}\@.
A \gls{not:set} is a collection of objects.
Some sample code is shown in. This uses the function \gls{fn.sqrt}.\glsadd{sqrt}
A \emph{\gls{[textbf]tautology}} is a proposition that is always true for any 
value of its variables.
A \emph{\gls{[textbf]contradiction}} is a proposition that is always false for any value of its variables.
%\include{chapters/M&M}

\section{Protein extraction from gastric biopsy material}
Buffer preparation:\\
\begin{itemize}
\item Tris 62.5mM
\item program
\end{itemize}
\chapter{Results}
\label{ch:results}
\chapter{Conclusions}
\label{ch:conc}
\backmatter
\printterms[database=glossary]
\printterms[database=acronym]
\printterms[database=notation]
%\printindex
% A glossary and list of acronyms may go here
% or may go in the front matter after the abstract.
% The bibliography will go here
\printbibliography[title={References}] % change from bibliography to references
\printterms[database=index]
\end{document}